\documentclass[]{article}

% Import packages
\usepackage[fleqn]{amsmath}
\usepackage{mathtools}
\usepackage{graphicx}
\usepackage[export]{adjustbox}
\usepackage{tikz}
\usepackage{xcolor}

\usetikzlibrary{positioning,shapes,arrows}

% Page initialization
\setlength{\topmargin}{-.3 in}
\setlength{\oddsidemargin}{0in}
\setlength{\evensidemargin}{0in}
\setlength{\textheight}{9.in}
\setlength{\textwidth}{6.5in}
\pagestyle{empty}

%==============================================================%
%------------------------START-DOCUMENT------------------------%
%==============================================================%
\begin{document}

%----------------------------HEADER----------------------------%

\begin{center}
    {\Large AERO 422 Homework \#2}\\ % Homework number
    \vspace{0.2 cm}
    Instructor: Vedang Deshpande\\ % Instructor name
    \vspace{0.2 cm}
    Due: September 22, 2021 at 12:40p.m.\\ % Due date
    \vspace{0.2 cm}
    Fall 2021\\ % semester
    \vspace{0.2 cm}
    (\textbf{25 Points})\\
\end{center}

\vspace{0.2 cm}

\begin{enumerate}

    %---------------------------PROBLEM-1--------------------------%
    \item Consider the function $f(t)=te^{2t}\sin{3t}$
    \begin{enumerate}
        \item (\textbf{2 points}) Find the Laplace transform using the table. Mention which entries from the table are being used.\\
        \textcolor{blue}{
        Refer to the following properties from the Laplace table
        \begin{flalign*}
            \mathcal{L}\left\{t^n f(t)\right\} = (-1)^n \frac{d^n F(s)}{ds^n} \quad (7), \qquad
            \mathcal{L}\left\{e^{at} \sin{kt}\right\} = \frac{k}{(s-a)^2 + k^2} \quad (22)
        \end{flalign*}
        Let $f(t)=e^{2t}\sin{3t}$. Apply $(7)$ to get
        \begin{flalign*}
            \mathcal{L}\left\{te^{2t}\sin{3t}\right\} &= \mathcal{L}\left\{tf(t)\right\}\\
            &= (-1) \frac{d F(s)}{ds}\\
            &= (-1) \frac{d \left[\mathcal{L}\left\{e^{2t} \sin{3t}\right\}\right]}{ds}
        \end{flalign*}
        Apply $(22)$ and simplify using the quotiet rule.
        \begin{flalign*}
        \mathcal{L}\left\{te^{2t}\sin{3t}\right\} &= (-1) \frac{d}{ds} \left\{\frac{3}{(s-2)^2+3^2}\right\}\\
        &= (-1) \frac{-3\cdot 2(s-2)}{((s-2)^2+9)^2}\\
        &= \frac{6s-12}{((s-2)^2+9)^2}
        \end{flalign*}
        }

        \item (\textbf{1 point}) Can we use the F.V.T. to determine $f(\infty)$? Why or why not?\\
        \textcolor{blue}{
        For the Final Value Theroem to apply, $f(t)$ must approach a finite value in the limit as $t$ approaches $\infty$. Upon inspection, we see that $f(t)=te^{2t}\sin{3t}$ diverges as $t \rightarrow \infty$, and thus does not approach a finite value.\\\\
        Another way to arrive at this conclusion is to look at the poles of the function. If a pole exists in the right-half plane, then the function will diverge. If the poles are puely imaginary, then the function will oscillate and will not approach a finite value in the limit as $t \rightarrow \infty$. Therefore, if there are no poles in the right-half plane, then at least one pole must lie in the left-half plane in order for the function to converge to a steady-state finite value.\\\\
        The poles of the transfer function $$F(s)=\frac{6s-12}{((s-2)^2+9)^2}$$ are $s=2+j3$, $2-j3$, $2+j3$, $2-j3$, which are all in the right-half plane, implying that this function is unstable.\\\\
        Therefore, the Final Value Theorem cannot be applied to determine $f(\infty)$.
        }
    \end{enumerate}
    \vspace{0.4 cm}


    %---------------------------PROBLEM-2--------------------------%
    \item Find the inverse Laplace transform using the table and partial fraction expansion. Show your work.
    \begin{enumerate}
        \item (\textbf{2 points}) $$F(s)=\frac{s+10}{s^2+2s+10}$$
        \textcolor{blue}{
            Refer to the following properties from the Laplace table
        \begin{flalign*}
            \mathcal{L}\left\{e^{at} \sin{kt}\right\} = \frac{k}{(s-a)^2 + k^2} \quad (22), \qquad
            \mathcal{L}\left\{e^{at} \cos{kt}\right\} = \frac{s-a}{(s-a)^2 + k^2} \quad (23)
        \end{flalign*}
        By completing the square, $F(s)$ can be re-written as follows:
        \begin{flalign*}
            F(s) &= \frac{s+10}{s^2+2s+10}\\
            &= \frac{s+10}{(s+1)^2+9}\\
            &= \frac{s+1}{(s+1)^2+3^2} + \frac{9}{(s+1)^2+3^2}\\
            &= \frac{s+1}{(s+1)^2+3^2} + \frac{3^2}{(s+1)^2+3^2}
        \end{flalign*}
        Take the inverse Laplace transform of $F(s)$ by using $(22)$ and $(23)$.
        \begin{flalign*}
            f(t) &= \mathcal{L}^{-1} \left\{F(s)\right\}\\
            &= \mathcal{L}^{-1} \left\{\frac{s+1}{(s+1)^2+3^2}\right\} + 3 \cdot \left\{\frac{3}{(s+1)^2+3^2}\right\}\\
            &= e^{-t} \cos{3t} + 3e^{-t} \sin{3t}\\
            &= e^{-t} \left(\cos{3t} + 3 \sin{3t}\right)
        \end{flalign*}
        }

        \item (\textbf{3 points}) $$F(s)=\frac{s^2+1}{s(s-1)^3}$$
        \textcolor{blue}{
        Refer to the following properties from the Laplace table
        \begin{flalign*}
            \mathcal{L}\left\{\frac{1}{s}\right\} = 1(t) \quad (1), \qquad
            \mathcal{L}\left\{\frac{1}{s-a}\right\} = e^{at} \quad (15), \qquad
            \mathcal{L}\left\{t^ne^{at}\right\} = \frac{n!}{(s-a)^{n+1}} \quad (21)
        \end{flalign*}
        \begin{flalign*}
            F(s) = \frac{s^2+1}{s(s-1)^3} \quad \rightarrow \quad \text{Poles at } s_1=0 \text{ and } s_2=1 \text{ (with multiplicity 3)}
        \end{flalign*}
        The partial fraction expansion takes the form
        \begin{flalign*}
            F(s) = \frac{A}{s} + \frac{B}{s-1} + \frac{C}{(s-1)^2} + \frac{D}{(s-1^3)}
        \end{flalign*}
        Reduce this form of $F(s)$ into a single rational function to get
        \begin{flalign*}
            F(s) = \frac{(A+B+C)s^3 + (-3A-2B-C)s^2 + (3A+B+D)s - A}{s(s-1)^3}
        \end{flalign*}
        Comparing this form of $F(s)$ to the original form yields a set of 4 equations with 4 unknowns as
        \begin{flalign*}
            A+B+C &= 0\\
            -3A-2B-C &= 1\\
            3A+B+D &= 0\\
            -A &= 1
        \end{flalign*}
        The solution to these equations is $A=-1$, $B=1$, $C=0$, and $D=2$. Then the partial fraction expansion of $F(s)$ is
        \begin{flalign*}
            F(s) = - \frac{1}{s} + \frac{1}{s-1} + \frac{2}{(s-1)^3}
        \end{flalign*}
        Apply $(1)$, $(15)$, $(21)$ to find $f(t)$.
        \begin{flalign*}
            f(t) &= \mathcal{L}^{-1} \left\{F(s)\right\}\\
            &= - \mathcal{L}^{-1} \left\{\frac{1}{s}\right\} + \mathcal{L}^{-1} \left\{\frac{1}{s-1}\right\} + \mathcal{L}^{-1} \left\{\frac{2}{(s-1)^3}\right\}\\
            &= -1(t) + e^t + t^2e^t, \quad t \geq 0
        \end{flalign*}
        }
    \end{enumerate}

    %---------------------------PROBLEM-3--------------------------%
    \item A given system is found to have a transfer function that is
    $$\frac{Y(s)}{R(s)}=\frac{10(s+2)}{s^2+8s+15}$$
    \begin{enumerate}
        \item (\textbf{3 points}) Using partial fractions, determine $y(t)$ when $r(t)$ is a unit step input. Show your work.\\
        \textcolor{blue}{Solution in other document (Problem 3b)}
        \item (\textbf{1 point}) Can we use F.V.T. to find $y(\infty)$? If the answer is yes, apply F.V.T. If not, explain why.\\
        \textcolor{blue}{Solution in other document (Problem 3a)}
    \end{enumerate}
    \vspace{0.4 cm}

    %---------------------------PROBLEM-4--------------------------%
    \item
    \begin{enumerate}
        \item (\textbf{3 points}) Using the convolution integral, find the step response of the system whose impulse response is given below
        \[
            h(t)=
            \begin{cases}
                te^{-t} & t \geq 0\\
                0       & t < 0
            \end{cases}
        \]
        \textcolor{blue}{
        There are only two cases to consider for this problem.\\\\
        Case (a): for the case $t \leq 0$, the situation is illustrated in Figure 1 as part (c). There is no overlap between the two functions $\left(u(t-\tau) \text{ and } h(\tau)\right)$ so the output is zero.
        \begin{flalign*}
            y_1(t) = 0, \quad t \leq 0
        \end{flalign*}
        Case (b): for the case $t \geq 0$, the situation is displayed in Figure 1 as part (d). In this figure, the unit step function is denoted as $u(t)$. The output of the system is given by
        \begin{flalign*}
            y_2(t) &= \int_{0}^{t} h(\tau) \cdot u(t-\tau) \,d\tau\\
            &= \int_{0}^{t} (\tau e^{-\tau})(1) \,d\tau  \quad \rightarrow \quad \text{integration by parts}\\
            &= 1-(t+1)e^-t, \quad t \geq 0
        \end{flalign*}
        }

        \item (\textbf{2 points}) Now use the Laplace transform table and partial fraction expansion to find $y(t)$.\\
        \textcolor{blue}{
        Refer to the following properties from the Laplace table
        \begin{flalign*}
            \mathcal{L}\left\{\frac{1}{s}\right\} = 1(t) \quad (1), \qquad
            \mathcal{L}\left\{\frac{1}{s-a}\right\} = e^{at} \quad (15), \qquad
            \mathcal{L}\left\{te^{at}\right\} = \frac{1}{(s-a)^2} \quad (20)
        \end{flalign*}
        Convolution in time-domain is equivalent to multiplication in frequency-domain.
        \begin{flalign*}
            Y(s) &= H(s) \cdot U(s)\\
            &= \mathcal{L} \left\{te^{-t}\right\} \cdot \mathcal{L} \left\{1\right\}\\
            &= \frac{1}{(s+1)^2} \cdot \frac{1}{s}
        \end{flalign*}
        The partial fraction expansion of this transfer function takes the form
        \begin{flalign*}
            Y(s) = \frac{A}{s} + \frac{B}{s+1} + \frac{C}{(s+1)^2}
        \end{flalign*}
        Reduce this form of $F(s)$ into a single rational function.
        \begin{flalign*}
            Y(s) &= \frac{A(s+1)^2 + Bs(s+1) + Cs}{s(s+1)^2}\\
            &= \frac{(A+B)s^2 + (2A+B+C)s + A}{s(s+1)^2}
        \end{flalign*}
        Comparing this form of $F(s)$ to the original form yields a set of 3 equations with 3 unknowns as
        \begin{flalign*}
            A+B &= 0\\
            2A+B+C &= 0\\
            A &= 1
        \end{flalign*}
        The solution to these equations is $A=1$, $B=-1$, and $C=-1$. Then the partial fraction expansion of $F(s)$ is
        \begin{flalign*}
            F(s) = \frac{1}{s} - \frac{1}{s+1} - \frac{1}{(s+1)^2}
        \end{flalign*}
        Apply $(1)$, $(15)$, $(20)$ to find $f(t)$.
        \begin{flalign*}
            f(t) &= \mathcal{L}^{-1} \left\{F(s)\right\}\\
            &= - \mathcal{L}^{-1} \left\{\frac{1}{s}\right\} - \mathcal{L}^{-1} \left\{\frac{1}{s+1}\right\} - \mathcal{L}^{-1} \left\{\frac{1}{(s+1)^2}\right\}\\
            &= 1 - e^{-t} - te^{-t}, \quad t \geq 0\\
            &= 1-(t+1)e^-t, \quad t \geq 0
        \end{flalign*}
        Note this solution is identical to the solution in part a), proving that convolution in time-domain is equivalent to multiplication in frequency-domain.
        }\vspace{0.4 cm}


        \item (\textbf{2 points}) Apply I.V.T. and F.V.T. (if applicable) to find $y(0)$ and $y(\infty)$.\\
        \textcolor{blue}{
        By inspection, both Initial Value Theorem and Final Value Theorem can be applied because $y(t)$ results in finite values for $t \rightarrow 0$ and $t \rightarrow \infty$.\\\\
        Initial Value Theorem:\\
        \begin{flalign*}
            \lim_{t \to 0} f(t) &= \lim_{s \to \infty} s \cdot Y(s)\\
            &= \lim_{s \to \infty} s \cdot \frac{1}{(s+1)^2} \cdot \frac{1}{s}\\
            &= \lim_{s \to \infty} \frac{1}{(s+1)^2}\\
            &= 0
        \end{flalign*}
        Final Value Theorem:\\
        \begin{flalign*}
            \lim_{t \to \infty} f(t) &= \lim_{s \to 0} s \cdot Y(s)\\
            &= \lim_{s \to 0} s \cdot \frac{1}{(s+1)^2} \cdot \frac{1}{s}\\
            &= \lim_{s \to 0} \frac{1}{(s+1)^2}\\
            &= 1
        \end{flalign*}
        }
    \end{enumerate}
    % \vspace{0.4 cm}

    %---------------------------PROBLEM-5--------------------------%
    \item For each of the following block diagrams, reduce the block diagram to find $T(s)$, where $T(s)$ is defined by $Y(s)=T(s)R(s)$.
    
    % define the blocks
    \tikzstyle{block} = [draw, fill=black!10, rectangle, minimum height=2.5em, minimum width=4em]
    \tikzstyle{sum} = [draw, circle, node distance=1cm]
    \tikzstyle{input} = [coordinate]
    \tikzstyle{output} = [coordinate]
    
    \begin{enumerate}
        \item (\textbf{1.5 points}) \\
        \textcolor{blue}{Solution in other document (Problem 4a)}
        \leavevmode\vadjust{\vspace{-\baselineskip}}\newline \\ \\% Part A
        % Create image of block diagram
        \begin{tikzpicture}[auto, node distance=2cm,>=latex'][h]

            % Place the blocks in desired locations
            \node [input, name=input] {};
            \node [sum, right of=input, node distance=1.5cm] (sum1) {};
            \node [block, right of=sum1, node distance=4cm] (G) {$G(s)$};
            \node [block, above of=G, node distance=1.5cm] (H1) {$H_1(s)$};
            \node [block, below of=G, node distance=1.5cm] (H2) {$H_2(s)$};
            \node [sum, right of=G, node distance=4cm] (sum2) {};
            \node [output, right of=sum2, node distance=1.5cm] (output) {};

            % Once the nodes are placed, connecting them is easy. 
            \draw [draw,->] (input) -- node[left of=input, node distance=1.2cm] {$R(s)$} node[pos=0.94] {$+$} (sum1);
            \draw [->] (sum1) -- node[name=error] {} (G);
            \draw [->] (error.east|-G) |- (H1);
            \draw [->] (G) -- node[name=y] {} node[pos=0.97,below] {$+$} (sum2);
            \draw [->] (y) |- (H2);
            \draw [->] (H2) -| node[pos=0.95] {$-$} (sum1);
            \draw [->] (H1) -| (sum2) node[pos=0.95,left] {$+$};
            \draw [draw,->] (sum2) -- node[right of=input, node distance=1.2cm] {$Y(s)$} (output);

        \end{tikzpicture}
        \vspace{0.6 cm}

        \item (\textbf{3 points}) \\
        \textcolor{blue}{Solution in other document (Problem 4b)}
        \leavevmode\vadjust{\vspace{-\baselineskip}}\newline \\ \\ % Part B
        % Create image of block diagram
        \begin{tikzpicture}[auto, node distance=2cm,>=latex'][h]

            % Place the blocks in desired locations
            \node [input, name=input] {};
            \node [sum, right of=input, node distance=4cm] (sum1) {};
            \node [block, above of=sum1, node distance=1.5cm] (G1) {$G_1(s)$};
            \node [sum, right of=sum1, node distance=2cm] (sum2) {};
            \node [block, right of=sum2, node distance=2.5cm] (G2) {$G_2(s)$};
            \node [block, below of=G2, node distance=1.5cm] (H1) {$H_1(s)$};
            \node [sum, left of=H1, node distance=2.5cm] (sum3) {};
            \node [block, below of=H1, node distance=1.5cm] (H2) {$H_2(s)$};
            \node [output, right of=G2, node distance=4cm] (output) {};

            % Once the nodes are placed, connecting them is easy. 
            \draw [->] (input) -- node[name=r] {} node[left of=input, node distance=2.5cm] {$R(s)$} node[pos=0.97] {$+$} (sum1);
            \draw [->] (r.east|-sum1) |- (G1);
            \draw [->] (sum1) -- node[pos=0.95,below] {$+$} (sum2);
            \draw [->] (G1) -| node[pos=0.97,left] {$+$} (sum2);
            \draw [->] (sum2) -- (G2);
            \draw [->] (G2) -- node[name=y] {} node[right of=input, node distance=2.2cm] {$Y(s)$} (output);
            \draw [->] (y) |- (H1);
            \draw [->] (y) |- (H2);
            \draw [->] (H1) -- node[pos=0.95,above] {$+$} (sum3);
            \draw [->] (H2) -| node[pos=0.97,right] {$-$} (sum3);
            \draw [->] (sum3) -| node[pos=0.97,left] {$-$} (sum1);

        \end{tikzpicture}
        \vspace{0.6 cm}

        \item (\textbf{1.5 points}) \\
        \textcolor{blue}{Solution in other document (Problem 4c)}
        \leavevmode\vadjust{\vspace{-\baselineskip}}\newline \\ \\ % Part C
        % Create image of block diagram
        \begin{tikzpicture}[auto, node distance=2cm,>=latex'][h]

            % Place the blocks in desired locations
            \node [input, name=input] {};
            \node [block, right of=input, node distance=4cm] (G1) {$G_1(s)$};
            \node [sum, right of=G1, node distance=2.25cm] (sum1) {};
            \node [block, right of=sum1, node distance=2.25cm] (G2) {$G_2(s)$};
            \node [sum, right of=G2, node distance=2.25cm] (sum2) {};
            \node [output, right of=sum2, node distance=1.75cm] (output) {};

            % Once the nodes are placed, connecting them is easy. 
            \draw [draw,->] (input) -- node[left of=input, node distance=2.25cm] {$R(s)$} node[name=r] {} node[name=j1, below of=r, node distance=1.5cm] {} (G1);
            \draw [->] (G1) -- node[pos=0.92,above] {$+$} (sum1);
            \draw (r) |- (j1.center);
            \draw [->] (j1.center) -| node[pos=0.95,left] {$+$} (sum1);
            \draw [->] (sum1) -- (G2);
            \draw [->] (G2) -- node[pos=0.92,above] {$+$} (sum2);
            \draw [->] (j1.center) -| node[pos=0.95,left] {$+$} (sum2);
            \draw [draw,->] (sum2) -- node[right of=input, node distance=1.3cm] {$Y(s)$} (output);

        \end{tikzpicture}
        \vspace{0.6 cm}

    \end{enumerate}

\end{enumerate}

% insert image
\begin{figure}[h]
    \includegraphics[scale=0.53,center]{AERO_422_HW2_image.PNG}
    \caption{Convolution integral (reference for problem 4)}
\end{figure}

\end{document}


