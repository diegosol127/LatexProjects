\documentclass[]{article}

% Import packages
\usepackage[fleqn]{amsmath}
\usepackage{pgfplots}
\usepackage{tikz}
\usepackage{xcolor}

\usetikzlibrary{calc,math}
\pgfplotsset{compat=1.8}

% Page initialization
\setlength{\topmargin}{-.3 in}
\setlength{\oddsidemargin}{0in}
\setlength{\evensidemargin}{0in}
\setlength{\textheight}{9.in}
\setlength{\textwidth}{6.5in}
\pagestyle{empty}

%==============================================================%
%------------------------START-DOCUMENT------------------------%
%==============================================================%
\begin{document}

%----------------------------HEADER----------------------------%

\begin{center}
    {\Large AERO 422 Homework \#1}\\ % Homework number
    \vspace{0.2 cm}
    Instructor: Vedang Deshpande\\ % Instructor name
    \vspace{0.2 cm}
    Due: September 15, 2021 at 12:40p.m.\\ % Due date
    \vspace{0.2 cm}
    Fall 2021\\ % semester
    \vspace{0.2 cm}
    (\textbf{20 Points})\\
\end{center}

\vspace{0.2 cm}

\begin{enumerate}

    %---------------------------PROBLEM-1--------------------------%
    \item (\textbf{5 points}) For the following rational function, i) determine the poles and zeros, ii) plot the pole-zero map using MATLAB (or another appropriate alternative), and iii) find the magnitude and phase at $s=-1+j$.
    $$F(s)=\frac{6(s^2+2)}{s(s^2+2s+5)}$$
    \textcolor{red}{Solution in other document.}
    \vspace{0.2 cm}

    %---------------------------PROBLEM-2--------------------------%
    \item
    \begin{enumerate}
        \item (\textbf{2 points}) Using the \textit{definition} of the Laplace transform, find $F(s)=\mathcal{L}\{f(t)\}$ for $f(t)=e^{-at}\cdot1(t)$.
        \textcolor{red}{Solution in other document.}

        \item (\textbf{3 points}) Using the Laplace table and its properties, find the Laplace transform of $f(t)$ shown in the figure.
        \begin{center}
        \begin{tikzpicture}
            % Axis
            \begin{axis}[
                % Axis
                axis x line = middle,
                axis y line = middle,
                ylabel = $f(t)$,
                xlabel = $t$,
                xtick = {0,1},
                ytick = {0,1},
                xmax = 3,
                ymax = 2,
                xmin = -1,
                ymin = -0.25
                ]
                % Plots
                \addplot[thick,domain= -1:0] {0};
                \addplot[thick,domain=  0:1] {x};
                \addplot[thick,domain=  1:3] {1};
                \draw[thick,dashed] (axis cs:1,1) -- (axis cs:1, 0);
                \draw[thick,dashed] (axis cs:1,1) -- (axis cs:0, 1);
            \end{axis}
        \end{tikzpicture}
        \end{center}
        \textit{Hint: Express $f(t)$ in terms of simple components like ramp the function and unit step function.}

        \textcolor{red}{
        Solution:\\\\
        Express the piecewise function in terms of unit step function and ramp function. Simplify and then apply the Laplace transform to both sides.
        \begin{flalign*}
            f(t) &= t \cdot 1(t) - t \cdot 1(t-1) + 1(t-1)\\
            &= t \cdot 1(t) + (1-t) \cdot 1(t-1)\\
            &= t \cdot 1(t) - (t-1) \cdot 1(t-1)
        \end{flalign*}
        For the second term, apply this rule from the Laplace table: $\mathcal{L}\{f(t-a) \cdot 1(t-a)\} = e^{-as}F(s)$.
        \begin{flalign*}            
            \mathcal{L}\{f(t)\} = \mathcal{L}\{t \cdot 1(t)\} - \mathcal{L}\{(t - 1) \cdot 1(t-1)\}
        \end{flalign*}
        For the first term, recall that $1(t) = 1$ when $t \geq 0$, which is also the domain of the Laplace transform.
        \begin{flalign*}
            F(s) &= \mathcal{L}\{t\} - \mathcal{L}\{(t - 1) \cdot 1(t-1)\}\\
            &= \frac{1}{s^2} - e^{-s} \cdot \mathcal{L}\{t\}\\
            &= \frac{1}{s^2} - \frac{e^{-s}}{s^2}
        \end{flalign*}
        }
    \end{enumerate}
    % \vspace{0.4 cm}

    %---------------------------PROBLEM-3--------------------------%
    \item (\textbf{6 points}) For each of the following differential equations with specified initial conditions, use the properties of the Laplace transform to solve for $X(s)$, where $X(s)=\mathcal{L}\{x(t)\}$.
    \begin{enumerate}
        \item $\Ddot{x}(t)+2\Dot{x}(t)+5x(t)=3\cdot1(t),\hspace{0.2cm} x(0^+)=0,\hspace{0.1cm} \Dot{x}(0^+)=0$
        \item $\Ddot{x}(t)+2\zeta\omega_n\Dot{x}(t)+\omega_n^2x(t)=0,\hspace{0.2cm} x(0^+)=a,\hspace{0.1cm} \Dot{x}(0^+)=b$
        \item $\Dot{x}(t)+ax(t)=A\sin{\omega t},\hspace{0.2cm} x(0^+)=b$
    \end{enumerate}
    \textcolor{red}{Solutions in other document.}
    \vspace{0.4 cm}

    %---------------------------PROBLEM-4--------------------------%
    \item (\textbf{4 points}) Assuming that $\{f(t),F(s)\}$ are a Laplace transform pair, where
    $$F(s)=\frac{3s+1}{s^2+s+1},$$
    determine the values of $f(0^+)$, $\Dot{f}(0^+)$, and $f(\infty)$.\vspace{0.2 cm}\\
    \textit{Hint: Apply the initial value theorem approach to} $\Ddot{f}(t)$.
    
    \textcolor{red}{
    Solutions to $f(0^+)$ and $\dot{f}(0^+)$ in other document.\\\\
    Solution to $f(\infty)$:\\\\
    Apply Initial Value Theorem.
    \begin{flalign*}
        \lim_{t \to \infty} f(t) &= \lim_{s \to 0} s \cdot F(s)\\
        &= \lim_{s \to 0} s \cdot \frac{3s+1}{s^2+s+1}\\
        &= \lim_{s \to 0} \frac{3s^2+s}{s^2+s+1}\\
        &= \frac{0}{1}\\
        &= 0
    \end{flalign*}
    }
    \vspace{0.4 cm}

\end{enumerate}

\end{document}


