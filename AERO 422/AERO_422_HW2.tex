\documentclass[]{article}

% Import packages
\usepackage{mathtools}
\usepackage{graphicx}
\usepackage{wrapfig}
\usepackage{tikz}

\usetikzlibrary{shapes,arrows}

% Page initialization
\setlength{\topmargin}{-.3 in}
\setlength{\oddsidemargin}{0in}
\setlength{\evensidemargin}{0in}
\setlength{\textheight}{9.in}
\setlength{\textwidth}{6.5in}
\pagestyle{empty}

%==============================================================%
%------------------------START-DOCUMENT------------------------%
%==============================================================%
\begin{document}

%----------------------------HEADER----------------------------%

\begin{center}
    {\Large AERO 422 Homework \#2}\\ % Homework number
    \vspace{0.2 cm}
    Instructor: Vedang Deshpande\\ % Instructor name
    \vspace{0.2 cm}
    Due: [TBD] at 12:40p.m.\\ % Due date
    \vspace{0.2 cm}
    Fall 2021\\ % semester
    \vspace{0.2 cm}
    (\textbf{TBC Points})\\
\end{center}

\vspace{0.2 cm}

\begin{enumerate}

    %---------------------------PROBLEM-1--------------------------%
    \item Consider the function $f(t)=te^{2t}\sin{3t}$
    \begin{enumerate}
        \item (\textbf{2 points}) Find the Laplace transform using the table. Mention which entries from the table are being used.
        \item (\textbf{1 point}) Can we use the F.V.T. to determine $f(\infty)$? Why or why not?
    \end{enumerate}
    \vspace{0.4 cm}


    %---------------------------PROBLEM-2--------------------------%
    \item Find the inverse Laplace transform using the table and partial fraction expansion. Show your work.
    \begin{enumerate}
        \item (\textbf{2 points}) $$F(s)=\frac{s+10}{s^2+2s+10}$$
        \item (\textbf{3 points}) $$F(s)=\frac{s^2+1}{s(s-1)^3}$$
    \end{enumerate}
    \vspace{0.4 cm}

    %---------------------------PROBLEM-3--------------------------%
    \item A given system is found to have a transfer function that is
    $$\frac{Y(s)}{R(s)}=\frac{10(s+2)}{s^2+8s+15}$$
    \begin{enumerate}
        \item (\textbf{3 points}) Determine, by hand, $y(t)$ when $r(t)$ is a unit step input. Show your work.
        \item (\textbf{2 points}) Can we use F.V.T. to find $y(\infty)$? If the answer is yes, apply F.V.T. If not, explain why.
    \end{enumerate}
    \vspace{0.4 cm}

    %---------------------------PROBLEM-4--------------------------%
    \item    
    \begin{enumerate}
        \item (\textbf{3 points}) Using the convolution integral, find the step response of the system whose impulse response is given below
        \[
            T(n)=
            \begin{cases}
                te^{-t} & t \geq 0\\
                0       & t < 0
            \end{cases}
        \]
        \item (\textbf{2 points}) Now use the Laplace transform table and partial fraction expansion to find $y(t)$.
        \item (\textbf{2 points}) Apply I.V.T. and F.V.T. (if applicable) to find $y(0)$ and $y(\infty)$.
    \end{enumerate}
    \vspace{0.4 cm}

    \item For each of the following block diagrams, reduce the block diagram to find $T(s)$, where $T(s)$ is defined by $Y(s)=T(s)R(s)$.
    \begin{enumerate}
        \item 
    \end{enumerate}

\end{enumerate}

\begin{figure}
    \includegraphics[scale=0.25]{HW2_image.PNG} % width=\linewidth
    \label{fig:convolution}
\end{figure}



\tikzstyle{block} = [draw, fill=blue!20, rectangle, 
    minimum height=2.5em, minimum width=4em]
\tikzstyle{sum} = [draw, fill=blue!20, circle, node distance=1cm]
\tikzstyle{input} = [coordinate]
\tikzstyle{output} = [coordinate]

% The block diagram code is probably more verbose than necessary
\begin{tikzpicture}[auto, node distance=2cm,>=latex']
    % We start by placing the blocks
    \node [input, name=input] {};
    \node [sum, right of=input] (sum) {};
    \node [block, right of=sum] (G) {$G(s)$};
    % \node [block, above of=G, node distance=3cm] (H1) {$H_1(s)$}
    \node [block, right of=G, node distance=3cm] (H2) {$H_2(s)$};
    % We draw an edge between G and H2 block to 
    % calculate the coordinate u. We need it to place the measurement block. 
    \draw [->] (G) -- node[name=u] {$u$} (H2);
    \node [output, right of=H2] (output) {};
    \node [block, below of=u] (measurements) {Measurements};

    % Once the nodes are placed, connecting them is easy. 
    \draw [draw,->] (input) -- node {$r$} (sum);
    \draw [->] (sum) -- node {$e$} (G);
    \draw [->] (H2) -- node [name=y] {$y$}(output);
    \draw [->] (y) |- (measurements);
    \draw [->] (measurements) -| node[pos=0.99] {$-$} 
        node [near end] {$y_m$} (sum);
\end{tikzpicture}

\end{document}


