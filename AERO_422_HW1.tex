\documentclass[]{book}

%These tell TeX which packages to use.
\usepackage{array,epsfig}
\usepackage{amsmath}
\usepackage{amsfonts}
\usepackage{amssymb}
\usepackage{amsxtra}
\usepackage{amsthm}
\usepackage{mathrsfs}
\usepackage{pgfplots}
\usepackage{tikz}
\usetikzlibrary{calc,math}

\pgfplotsset{compat=1.8}

%Pagination stuff.
\setlength{\topmargin}{-.3 in}
\setlength{\oddsidemargin}{0in}
\setlength{\evensidemargin}{0in}
\setlength{\textheight}{9.in}
\setlength{\textwidth}{6.5in}
\pagestyle{empty}



\begin{document}


\begin{center}
{\Large AERO 422 Homework \#1}\\
\vspace{0.2 cm}
Instructor: Vedang Deshpande\\ %You should put your name here
\vspace{0.2 cm}
Due: September 15, 2021 at 12:40p.m.\\
\vspace{0.2 cm}
Fall 2021\\
\vspace{0.2 cm}
(\textbf{20 Points})\\
% \vspace{0.2 cm}
% Due: DATE????? %You should write the date here.
\end{center}

\vspace{0.2 cm}


% \subsection*{Exercises for Section 1.1: Norm and Inner Product}

\begin{enumerate}

\item (\textbf{5 points}) For the following rational function, i) determine the poles and zeros, ii) plot the pole-zero map using MATLAB (or another appropriate alternative), and iii) find the magnitude and phase at $s=-1+j$.
$$F(s)=\frac{6(s^2+2)}{s(s^2+2s+5)}$$
\vspace{0.2 cm}

\item
\begin{enumerate}
    \item (\textbf{2 points)} Using the \textit{definition} of the Laplace transform, find $F(s)=\mathcal{L}\{f(t)\}$ for $f(t)=e^{-at}\cdot1(t)$.
    \item (\textbf{3 points}) Using the Laplace table and its properties, find the Laplace transform of $f(t)$ shown in the figure.
    \begin{center}
    \begin{tikzpicture}
        % Axis
        \begin{axis}[
            % Axis
            axis x line = middle,
            axis y line = middle,
            ylabel = $f(t)$,
            xlabel = $t$,
            xtick = {0,1},
            ytick = {0,1},
            xmax = 3,
            ymax = 2,
            xmin = -1,
            ymin = -0.25
            ]
            % Plots
            \addplot[thick,domain= -1:0] {0};
            \addplot[thick,domain=  0:1] {x};
            \addplot[thick,domain=  1:3] {1};
            \draw[thick,dashed] (axis cs:1,1) -- (axis cs:1, 0);
            \draw[thick,dashed] (axis cs:1,1) -- (axis cs:0, 1);
        \end{axis}
    \end{tikzpicture}
    \end{center}
    \textit{Hint: Express $f(t)$ in terms of simple components like ramp the function and unit step function.}
\end{enumerate}
\vspace{0.4 cm}

\item (\textbf{6 points}) For each of the following differential equations with specified initial conditions, use the properties of the Laplace transform to solve for $X(s)$, where $X(s)=\mathcal{L}\{x(t)\}$.
\begin{enumerate}
    \item $\Ddot{x}(t)+2\Dot{x}(t)+5x(t)=3\cdot1(t),\hspace{0.2cm} x(0^+)=0,\hspace{0.1cm} \Dot{x}(0^+)=0$
    \item $\Ddot{x}(t)+2\zeta\omega_n\Dot{x}(t)+\omega_n^2x(t)=0,\hspace{0.2cm} x(0^+)=a,\hspace{0.1cm} \Dot{x}(0^+)=b$
    \item $\Dot{x}(t)+ax(t)=A\sin{\omega t},\hspace{0.2cm} x(0^+)=b$
\end{enumerate}
\vspace{0.4 cm}

\item (\textbf{4 points}) Assuming that $\{f(t),F(s)\}$ are a Laplace transform pair, where
$$F(s)=\frac{3s+1}{s^2+s+1},$$
determine the values of $f(0^+)$, $\Dot{f}(0^+)$, and $f(\infty)$.\vspace{0.2 cm}\\
\textit{Hint: Apply the initial value theorem approach to} $\Ddot{f}(t)$.
\vspace{0.4 cm}

\end{enumerate}

\end{document}


